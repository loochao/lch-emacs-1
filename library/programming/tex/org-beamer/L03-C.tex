% Created 2011-05-19 Thu 03:51
\documentclass[10pt, t]{beamer}
\usepackage[utf8]{inputenc}
\usepackage[T1]{fontenc}
\usepackage{fixltx2e}
\usepackage{graphicx}
\usepackage{longtable}
\usepackage{float}
\usepackage{wrapfig}
\usepackage{soul}
\usepackage{textcomp}
\usepackage{marvosym}
\usepackage{wasysym}
\usepackage{latexsym}
\usepackage{amssymb}
\usepackage{hyperref}
\tolerance=1000
\usepackage{APC524b}
\providecommand{\alert}[1]{\textbf{#1}}

\title{C}
\author{robert lupton\thanks{LooChao@gmail.com}}
\date{23 september 2010}

\begin{document}

\maketitle

\begin{frame}
\frametitle{Outline}
\setcounter{tocdepth}{3}
\tableofcontents
\end{frame}

\section{Introduction}
\label{sec-1}
\subsection{types}
\label{sec-1_1}
\begin{frame}
\frametitle{Introduction}
\label{sec-1_1_1}

C was designed as a system programming language, to remove the necessity of writing operating systems in
assembler.  It's one of the large family of languages deriving from Algol.

\pause
You've seen ``hello world'' before:
CANNOT INCLUDE FILE src/hello.c
\pause

This example comes from ``The C Programming Language'' by Brian Kernighan and Dennis Ritchie (``K\&R'')
\end{frame}
\begin{frame}[fragile]
\frametitle{Declarations}
\label{sec-1_1_2}

 All variables must be declared before they can be used:
\begin{verbatim}
char c = 'a';
char *s = "I am a string";
double d = 1;
float f = 1.0f;
int i = 101010101;
long l = 10;
short s = 10;
\end{verbatim}
\pause
``string'' is spelt \code{char *} for reasons that I'll explain in a bit.
\pause
\code{short} is actually a shorthand for \code{short int} (\code{long} means \code{long int}).  
You can also define integral types (\code{char} and \code{int}) as being \code{signed} or \code{unsigned}. 
\pause One especially useful qualifier is \code{const}:
\begin{verbatim}
const unsigned long bad = 0xdeadbeef00000000;
\end{verbatim}

\pause An unqualified \code{int} is supposed to be the most efficient integral type on your machine, and is
what you'd generally use unless there was some reason not to. 
\end{frame}
\begin{frame}[fragile]
\frametitle{Declaring your own types}
\label{sec-1_1_3}

C allows you to use your own name for a type:
\begin{verbatim}
typedef unsigned short U16;
\end{verbatim}
\pause
Why would I want to do this?

\pause
CCD image data is typically created using a 16-bit A/D converter, so
the natural type for a single pixel is a 2-byte integer.

But C doesn't tell me how large an \code{int} is; I 
can find out (using \code{sizeof(int)}) but I don't want to have to change all my declarations when I move
to a new system.

\pause
Actually, these days, I could say:
\begin{verbatim}
#include <stdint.h>
typedef uint16_t U16;
\end{verbatim}
but even that only works if the processor actually has an unsigned 16-bit type.
\end{frame}
\begin{frame}[fragile]
\frametitle{Declarations}
\label{sec-1_1_4}

You can mix code and declarations:
\begin{verbatim}
int pwv = 0;                            // Precipitable Water Vapour
/*
 * Estimate the PWV based on the altitude
 */
...;
/*
 * Given that estimate, find a better value
 */
const int pwv0 = pwv;                   /* initial estimate of pwv */
\end{verbatim}
\pause
The ability to declare variables when they are first needed means that you can usually initialize them too;
when possible, make them \code{const}.
\end{frame}
\begin{frame}[fragile]
\frametitle{Scope}
\label{sec-1_1_5}

A variable's \emph{scope} is the part of the programme it may be referenced from;  in C, a variable's scope
is the nearest set of braces (\code{\{\}}), a \emph{block}.  If it isn't in a block, a variable is visible globally
(i.e. it'll show up when you run \code{nm} on your object file).  If this isn't what you wanted, you can:
 \pause

\begin{itemize}
\item Move it into a function --- remember, global variables should usually be avoided
\end{itemize}
 \pause

\begin{itemize}
\item Label it \code{static} which makes it only visible within the file
\end{itemize}
 \pause

\begin{itemize}
\item Decide that it really \emph{must} be globally visible, and declare it in a header file:
\begin{verbatim}
extern int nread;     // Number of times I've read from a file
\end{verbatim}
\end{itemize}

\pause
It is generally a good idea to declare a variable in as restricted a scope as possible (the ``No Globals'' rule 
is a special case of this one).
\end{frame}
\section{Program Structure}
\label{sec-2}
\subsection{\texttt{for}, \texttt{if}, \ldots{}}
\label{sec-2_1}
\begin{frame}
\frametitle{Control Flow}
\label{sec-2_1_1}


C has the usual constructs:

\begin{itemize}
\item \code{if(...) \{ ... \} else if(...) \{ ...\} else \{ ... \}}
\end{itemize}
 \pause

\begin{itemize}
\item \code{for(...) \{ ... \}}
\item \code{while \{ ... \}}
\item \code{do \{ ... \} while (...);}
\end{itemize}
 \pause

\begin{itemize}
\item \code{switch (...) \{ case XXX: ... break; ... \} }
\end{itemize}
 \pause

\begin{itemize}
\item \code{break}, \code{continue}
\end{itemize}
 \pause

\begin{itemize}
\item \code{goto}
\end{itemize}
\end{frame}
\begin{frame}[fragile]
\frametitle{If statements}
\label{sec-2_1_2}

\begin{verbatim}
if (i < 9) {
    printf("Hello");
} else {
    printf("Goodbye");
}
printf(" world\n");
\end{verbatim}
\pause
Legally, you can write this as
\begin{verbatim}
if (i < 9)
    printf("Hello");
else
    printf("Goodbye");

printf(" world\n");
\end{verbatim}
But we don't recommend it;
\pause  it's too easy to write
\begin{verbatim}
if (i < 9)
    printf("Hello");
else
    printf("Goodbye");
    printf(" and good riddance");

printf(" world\n");
\end{verbatim}
\end{frame}
\begin{frame}[fragile]
\frametitle{For loops}
\label{sec-2_1_3}

CANNOT INCLUDE FILE src/for.c
This only works with a compiler that supports the C99 standard (\code{cc --std=c99 ...})
\pause
This is almost equivalent to:
\begin{verbatim}
#include <stdio.h>

int
main()
{
     int i = 0;
     while (i != 10) {
          printf("Hello world\n");
          ++i;
     }

     return 0;
}
\end{verbatim}
\end{frame}
\begin{frame}[allowframebreaks]
\frametitle{Switch}
\label{sec-2_1_4}

CANNOT INCLUDE FILE src/printf.c
\end{frame}
\begin{frame}[fragile]
\frametitle{Switch}
\label{sec-2_1_5}

\begin{verbatim}
$ make printf && ./printf
$ cc -g -Wall -O3 --std=c99    printf.c   -o printf
a
b
c
:
<space>
<integer>
%
<space>
d
o
n
e
<newline>
\end{verbatim}
(N.b. The call was \code{checkFormat("abc: \%d\%\% done\\n");})
\end{frame}
\begin{frame}
\frametitle{Putting that together}
\label{sec-2_1_6}

CANNOT INCLUDE FILE src/switch.c
\end{frame}
\section{Functions}
\label{sec-3}
\subsection{subroutines, arrays, operators}
\label{sec-3_1}
\begin{frame}
\frametitle{Subroutines}
\label{sec-3_1_1}

Here is a function to add two numbers:

\snippetFile[C++]{src/c_examples.c}

\includeSnippet{add}
\pause
and here is one to multiply them:

\includeSnippet{mult}

\pause
So far, just like Fortran

\pause
\includeSnippet{square}

\pause
That's a little clumsier --- in Fortran (or python) you could have said \code{x**2}.  And 
note the extra include file, \file{math.h}.  In the old days you needed to link \code{-lm} too, but modern 
systems seem to be more forgiving.
\end{frame}
\begin{frame}
\frametitle{Recursion}
\label{sec-3_1_2}

Subroutines may be called recursively --- that is, they may call themselves, either directly or indirectly.
There is no limit except the capacity of your computer.
CANNOT INCLUDE FILE src/factorial.c
\end{frame}
\begin{frame}[fragile]
\frametitle{Recursion}
\label{sec-3_1_3}

That may not seem very interesting; it's easy enough to write a loop to calculate factorials.  

However, consider a routine \code{integrate(float a, float b, float (*func)(float x))} \footnote{\code{float (*func)(float x)} is C for ``a function \texttt{func} expecting a float and returning a float'' }

\pause

If I need to do a double integral, I can say something like:
\begin{verbatim}
static float yy;                        // current value of y

static float func(float x) {
   return sin(x)*cos(yy);
}

static float dfunc(float y) {
   yy = y;                              // pass y to func
   return integrate(0, y, func);
}

const double ans = integrate(1, 3, dfunc);
\end{verbatim}

to calculate
$$
\int_{x=0}^{y} \int_{y=1}^{3} \sin x \cos y \,  d\!x d\!y
$$
\end{frame}
\begin{frame}[fragile]
\frametitle{Complete Example}
\label{sec-3_1_4}

\begin{verbatim}
#include <stdio.h>
#include <math.h>

double integrate(const float a, const float b, float (*func)(float x)) {
      const int nstep = 1000;           // number of steps
      const float step = (b - a)/nstep;

      double ans = 0.0;
      float x = a;
      for (int i = 0; i != nstep; ++i) {
         ans += func(x); 
         x += step;
      }

      return step*ans;
}

static float yy;                        // current value of y

static float func(float x) {
   return sin(x)*cos(yy);
}

static float dfunc(float y) {
   yy = y;                              // pass y to func
   return integrate(0, y, func);
}

int main() {
   printf("ans = %g\n", integrate(1, 3, dfunc));
   return 0;
}
\end{verbatim}
\end{frame}
\begin{frame}[fragile]
\frametitle{Subroutine Arguments}
\label{sec-3_1_5}

Question: What does this do?
\begin{verbatim}
void triple(double x) {
   x *= 3;
}
...
double x = 1;
triple(x);
\end{verbatim}
\pause
Answer: wastes CPU cycles.

The function \code{triple} is passed a \emph{copy} of \code{x}, so \textbf{nothing} that \code{triple} does can 
affect the program that calls it.

\pause
The solution is to pass a \emph{pointer} to x:
\begin{verbatim}
void triple(double @\color{red}*@x) {
    @\color{red}*@x *= 3;
}
...
double x = 1;
triple(@\color{red}\&@x);
\end{verbatim}
\pause
Fortran \emph{always} passes arguments this way; a famous party trick when I was young was to say
\begin{verbatim}
call triple(1)
\end{verbatim}
\end{frame}
\begin{frame}[fragile]
\frametitle{Prototypes}
\label{sec-3_1_6}

It is critically important that subroutines' callers and callees agree about the number and
types of the arguments (the \emph{signature}).  C uses a \emph{prototype} to allow the compiler to check; given
\begin{verbatim}
void triple(double *x) {
    *x *= 3;
}
\end{verbatim}
we put the prototype \code{void triple(double *x);} in a header (``.h'') file and \code{#include} 
it in the file that defines \code{triple} and also whenever we call
\code{triple}. \footnote{Most compilers these days will complain if you don't do this. }

This isn't a lot of extra work and soon becomes second nature.
\end{frame}
\begin{frame}[fragile]
\frametitle{Arrays}
\label{sec-3_1_7}

Arrays are declared and subscripted using \code{[]}.
\begin{verbatim}
int ids[10];
\end{verbatim}
\pause
In C99, the dimension need \emph{not} be known at compile time:
\begin{verbatim}
void foo(const int n) {
    int ids[n];
    ...
    double x = ids[n/2];
}
\end{verbatim}
The index starts at \emph{0} (not \emph{1}, as in Fortran) as it specifies the distance from the start of the array.
\begin{verbatim}
printf("ID0: %d\n", ids[0]);
\end{verbatim}
\pause
This may make more sense when we discuss \emph{pointers}.
\end{frame}
\begin{frame}[fragile]
\frametitle{\texttt{n}-D Arrays}
\label{sec-3_1_8}

You can also define \texttt{n}-D arrays:
\begin{verbatim}
U16 data[4096][2048];
...
U16 const peak = data[y][x];            // the (x, y)th pixel
\end{verbatim}

The data is stored row-by-row so the \emph{last} index increases fastest if we access pixels in the order in which
they're stored (unlike Fortran arrays, in which the \emph{first} index varies fastest).

\pause
\texttt{n}-D arrays aren't as useful as you might think as they are passed to
subroutines as \emph{pointers}; for now all you need to know is that you \textbf{must} specify
all but the first dimension when passing an \texttt{n}-D array to a subroutine:
\begin{verbatim}
void debias(U16 data[][2048], const int nrow, const int ncol) {
   ...
}
\end{verbatim}
\pause (hmm, I seem to have typed \code{2048} twice; was that a good idea\ldots{}?)
\end{frame}
\begin{frame}
\frametitle{Operator Precedence}
\label{sec-3_1_9}

\newcommand{\pipe}{|}

\begin{center}
\begin{tabular}{ll}
\hline
 operator                                                                                                                                    &  associativity  \\
\hline
 \texttt{()} \texttt{[]} \texttt{->} \texttt{.}                                                                                              &  left to right  \\
 \texttt{!} \texttt{\textasciitilde{}} \texttt{++} \texttt{-}\texttt{-} \texttt{-} \texttt{(type)} \texttt{*} \texttt{\&} \texttt{sizeof()}  &  right to left  \\
 \texttt{*} \texttt{/} \texttt{\%}                                                                                                           &  left to right  \\
 \texttt{+} \texttt{-}                                                                                                                       &  left to right  \\
 \texttt{<}\texttt{<} \texttt{>}\texttt{>}                                                                                                   &  left to right  \\
 \texttt{<} \texttt{<=} \texttt{>} \texttt{>=}                                                                                               &  left to right  \\
 \texttt{==} \texttt{!=}                                                                                                                     &  left to right  \\
 \texttt{\&}                                                                                                                                 &  left to right  \\
 \texttt{\textasciicircum{}}                                                                                                                 &  left to right  \\
 \texttt{\pipe}                                                                                                                              &  left to right  \\
 \texttt{\&\&}                                                                                                                               &  left to right  \\
 \texttt{\pipe\pipe}                                                                                                                         &  left to right  \\
 \texttt{?:}                                                                                                                                 &  right to left  \\
 \texttt{=} \texttt{+=} \texttt{-=} etc.                                                                                                     &  right to left  \\
 \texttt{,}                                                                                                                                  &  left to right  \\
\hline
\end{tabular}
\end{center}
\end{frame}
\section{Macros}
\label{sec-4}
\subsection{\texttt{\#if}, \texttt{\#define}}
\label{sec-4_1}
\begin{frame}[fragile]
\frametitle{Macros and the C Pre-Processor (\emph{CPP})}
\label{sec-4_1_1}

C provides a simple macro processor which can be used to keep \emph{magic numbers} out of your
code[3]
instead of
\begin{verbatim}
char configFile[21];                    // name of configuration file
fgets(stdin, fileName, 20);
\end{verbatim}
\pause
you can write
\begin{verbatim}
#define FILE_LEN 20                     // maximum length for filename

char configFile[FILE_LEN + 1];          // name of configuration file
fgets(stdin, fileName, FILE_LEN);
\end{verbatim}
\pause writing all macros in CAPITALS is a common convention.
\end{frame}
\begin{frame}[fragile]
\frametitle{Conditional Compilation}
\label{sec-4_1_2}

In the bad old days, we wrote lots of code like:
\begin{verbatim}
#if defined(vms)
   return -1;
#elif defined(HAVE_SELECT) && !defined(USE_POLL)
   if(select(ncheck,&mask,(fd_set *)NULL,(fd_set *)NULL,
             (void *)NULL) != 0) {
        ...
   }
#else
  /* Use poll() */
#endif
\end{verbatim}
\pause
Fortunately, in these days of \href{http://www.opengroup.org/onlinepubs/009695399/}{Posix} (\texttt{IEEE 1003}; \texttt{ISO/IEC 9945}) there is much less need for of this sort of thing.
\pause 
The (or at least my) main use of the CPP is:
\begin{verbatim}
#define DEBUG 1
...
#if DEBUG
int niter = 0;
#endif
...
#if DEBUG
printf("niter %d\n", ++niter);
#endif
\end{verbatim}
\end{frame}
\begin{frame}[fragile]
\frametitle{Good uses for macros I}
\label{sec-4_1_3}

One standard use for macros is to prevent header files being parsed more than once
\begin{verbatim}
#if !defined(GREET_H)
#define GREET_H
/* Lots of stuff that should be processed only once */
#endif
\end{verbatim}

This use of a macro is known as an \emph{include guard}.
\end{frame}
\begin{frame}[fragile]
\frametitle{Good uses for macros II}
\label{sec-4_1_4}

CANNOT INCLUDE FILE src/macros.c
\pause
\begin{verbatim}
$ make macros && macros 1 2 3
cc -g -Wall --std=c99 macros.c -o macros
macros.c:009: Hello world!
macros.c:010: My name is macros and I was called with 3 arguments.
\end{verbatim}

\pause
Unfortunately, this code does not conform to the C99 standard (\texttt{ISO 9899:1999}):
\begin{verbatim}
$ cc -g -Wall --std=c99 --pedantic-errors macros.c -o macros
macros.c:3:29: error: ISO C does not permit named variadic macros
macros.c:7:30: error: ISO C99 requires rest arguments to be used
\end{verbatim}
\end{frame}
\begin{frame}
\frametitle{Good uses for macros II}
\label{sec-4_1_5}

A legal version is:
CANNOT INCLUDE FILE src/macros2.c
Note that we were forced to pass at least one argument (hence the \code{""} in the first call).
\end{frame}
\begin{frame}[fragile]
\frametitle{Bad uses for macros I}
\label{sec-4_1_6}

You can use the CPP to pretend that you're writing Algol, not C
\begin{verbatim}
#define IF      if(
#define THEN    ){
#define ELSE    } else {
#define ELIF    } else if (
#define FI      ;}

IF x == 0 THEN
   printf("zero\n");
ELIF x == 1 THEN
   printf("one\n");
ELSE
   printf("many\n");
FI
\end{verbatim}
\pause
These macros come from the original Bourne shell source code in Unix Version 7.
\end{frame}
\begin{frame}[fragile]
\frametitle{Bad uses for macros II}
\label{sec-4_1_7}

How about
\begin{verbatim}
#define MAX(a,b)        ((a)>(b)?(a):(b))
\end{verbatim}
This one also comes from Steve Bourne.

\pause
Why is this bad?  Consider
\begin{verbatim}
const double fg = MAX(funcs(x), gunks(x));
\end{verbatim}
\pause
\begin{verbatim}
double funcs(const double x) {
   static int x = 0.5;
   x = sin(x);
   return x;
}

static int i = 0;
double gunks(const double x) {
   return (x < ++i) ? 5.6 : 5.9;
}
\end{verbatim}
\pause
Both \code{funcs} and \code{gunks} are called twice; this is always inefficient, but in this case
 catastrophic as they maintain internal state.

\pause
You don't write code like \emph{that}, of course.  But \code{MAX(getchar(), 'a')} is just as bad.

\pause
C99 (and \CPP) have the \code{inline} keyword which removes some of the desire for macros like this.
\end{frame}
\section{Structs and Memory}
\label{sec-5}
\subsection{\texttt{struct}, the stack, \texttt{malloc}}
\label{sec-5_1}
\begin{frame}[fragile]
\frametitle{Structs}
\label{sec-5_1_1}

If you routinely write code like:
\begin{verbatim}
void printObjects(const int n, const int id[],
                  const float xcen[], const float ycen[],
                  const float flux[]);

#define NOBJECT 1000                    // Maximum number of objects

int id[NOBJECT];                        // Object IDs
float xcen[NOBJECT];                    // x-coordinate of centre
float ycen[NOBJECT];                    // y-coordinate of centre
float flux[NOBJECT];                    // object's flux
  
for (int i = 0; i < n; ++i) {
   id[i] = i;
   xcen[i] = ...;
   ycen[i] = ...;
   flux[i] = ...;
}

printObjects(n, id, xcen, ycen, flux);
\end{verbatim}
C and we can help.
\end{frame}
\begin{frame}[fragile]
\frametitle{Structs}
\label{sec-5_1_2}

A cleaner way to write this is:
\begin{verbatim}
struct Object {
   int id;                              // Object IDs
   float xcen;                          // x-coordinate of centre
   float ycen;                          // y-coordinate of centre
};

typedef struct Object Object;

void printObjects(const int n, const Object objs[]);

#define NOBJECT 1000                    // Maximum number of objects

Object objs[NOBJECT];                   // our objects

for (int i = 0; i < n; ++i) {
   objs[i].id = i;
   objs[i].xcen = ...;
   objs[i].ycen = ...;
}

printObjects(n, objs);
\end{verbatim}
Now if we need to add more fields (e.g. \code{float flux;}) we don't need to 
change \code{printObjects}' signature.

\pause
The top of this example (down to \code{#define NOBJECT}) would usually be put in a 
header file with \emph{include guards}
\end{frame}
\begin{frame}
\frametitle{Memory Allocation}
\label{sec-5_1_3}

You should be a little uneasy about \code{NOBJECT} in that last example.  A number of questions come to mind:

\begin{itemize}
\item How did I know that I only had 1000 objects?
\item What would I do if I had more?
\item Am I wasting space if I have less?
\end{itemize}
\pause
and also

\begin{itemize}
\item Where is the computer putting all those IDs and positions?
\end{itemize}
\end{frame}
\begin{frame}
\frametitle{Inside your job}
\label{sec-5_1_4}

A running program consists of a number of pieces:

\begin{itemize}
\item \pause \texttt{text}  Our instructions and read-only data
\item \pause \texttt{data}  Initialised global data
\end{itemize}

\begin{itemize}
\item \pause \texttt{bss}   Uninitialized global data
\item \pause \texttt{stack} Memory for variables in subroutines
\item \pause \texttt{heap}  Memory available to the programmer
\end{itemize}

\pause
All of these are mapped into a single (logical) section of RAM:


\begin{center}
\begin{tabular}{|*{8}{l|}}
\hline
 \texttt{0x0}  &  text  &  data  &  bss  &  heap  &  \hspace{1cm}  &  stack  \\
\hline
\end{tabular}
\end{center}



(\texttt{0x0} is how you write a hexadecimal number in C (or \CPP))
\end{frame}
\begin{frame}
\frametitle{Stack and Heap}
\label{sec-5_1_5}


It's traditional to think of the heap and stack as growing down and up respectively:


\begin{center}
\begin{tabular}{|l|l|}
\hline
 heap   &  $\downarrow$  \\
\hline
 \null  &                \\
\hline
 stack  &  $\uparrow$    \\
\hline
\end{tabular}
\end{center}
\end{frame}
\begin{frame}[fragile]
\frametitle{The Stack}
\label{sec-5_1_6}


What happens when I call a function? E.g.
\begin{verbatim}
double integrate(const float a, const float b, float (*func)(float x)) {
      const int nstep = 1000;           // number of steps
      const float step = (b - a)/nstep;
      ...
      ans += func(x);
}

double ans = integrate(1, 3, dfunc);
\end{verbatim}
\pause
We first push the values of \texttt{a}, \texttt{b}, and \texttt{func} onto the stack:


\begin{center}
\begin{tabular}{|c|}
\hline
 \texttt{<dfunc>}  \\
\hline
 \texttt{3.0}      \\
\hline
 \texttt{1.0}      \\
\hline
\hline
\end{tabular}
\end{center}



\pause
We then make space for \texttt{step} and \texttt{nstep}


\begin{center}
\begin{tabular}{|c|}
\hline
 \texttt{???}      \\
\hline
 \texttt{???}      \\
\hline
 \texttt{<dfunc>}  \\
\hline
 \texttt{3.0}      \\
\hline
 \texttt{1.0}      \\
\hline
\hline
\end{tabular}
\end{center}
\end{frame}
\begin{frame}
\frametitle{The Stack}
\label{sec-5_1_7}


We initialised \texttt{step} to \texttt{1000} and \texttt{nstep} to \texttt{(b - a)/nstep}, so:


\begin{center}
\begin{tabular}{|c|}
\hline
 \texttt{0.002}    \\
\hline
 \texttt{1000}     \\
\hline
 \texttt{<dfunc>}  \\
\hline
 \texttt{3.0}      \\
\hline
 \texttt{1.0}      \\
\hline
\hline
\end{tabular}
\end{center}


\pause

Within \texttt{integrate} we call \texttt{dfunc} with argument \texttt{x}; this pushes \texttt{x} (\texttt{1.0}) onto the stack:


\begin{center}
\begin{tabular}{|c|}
\hline
 \texttt{1.0}      \\
\hline
\hline
 \texttt{0.002}    \\
\hline
 \texttt{1000}     \\
\hline
 \texttt{<dfunc>}  \\
\hline
 \texttt{3.0}      \\
\hline
 \texttt{1.0}      \\
\hline
\hline
\end{tabular}
\end{center}
\end{frame}
\begin{frame}
\frametitle{The Stack}
\label{sec-5_1_8}


within \texttt{dfunc} we calculate \texttt{y} (\texttt{1.0}) and call \code{integrate(0, y, func)} resulting in:


\begin{center}
\begin{tabular}{|c|}
\hline
 \texttt{0.001}    \\
\hline
 \texttt{1000}     \\
\hline
 \texttt{<func>}   \\
\hline
 \texttt{1.0}      \\
\hline
 \texttt{0.0}      \\
\hline
\hline
 \texttt{1.0}      \\
\hline
\hline
 \texttt{0.002}    \\
\hline
 \texttt{1000}     \\
\hline
 \texttt{<dfunc>}  \\
\hline
 \texttt{3.0}      \\
\hline
 \texttt{1.0}      \\
\hline
\hline
\end{tabular}
\end{center}



And so on.
\end{frame}
\begin{frame}
\frametitle{The Stack}
\label{sec-5_1_9}


When \texttt{integrate} has calculated its return value, it puts it somewhere 
and \emph{pops} the stack:


\begin{center}
\begin{tabular}{|c|}
\hline
 \texttt{1.0}      \\
\hline
\hline
 \texttt{0.002}    \\
\hline
 \texttt{1000}     \\
\hline
 \texttt{<dfunc>}  \\
\hline
 \texttt{3.0}      \\
\hline
 \texttt{1.0}      \\
\hline
\hline
\end{tabular}
\end{center}



\pause

\texttt{dfunc} puts \emph{its} return value somewhere, and pops the stack:


\begin{center}
\begin{tabular}{|c|}
\hline
 \texttt{0.002}    \\
\hline
 \texttt{1000}     \\
\hline
 \texttt{<dfunc>}  \\
\hline
 \texttt{3.0}      \\
\hline
 \texttt{1.0}      \\
\hline
\hline
\end{tabular}
\end{center}



\pause 

And finally the outer call to \texttt{integrate} finishes, saves its value, pops the stack, and we're
back where we started.
\end{frame}
\begin{frame}
\frametitle{Subroutine Arguments Redux}
\label{sec-5_1_10}

It should now be clear that:

\begin{itemize}
\item A language can \textbf{only} pass variables by value if it wishes to support recursion
 [4]
\item Variables in subroutines are irretrievably lost when the routine returns
\item Uninitialized values may have any value
\item (Almost) all variables are stored somewhere in the process's memory
\end{itemize}

\pause The last bullet suggests a way to work around the first one: we can push the variable's \emph{address} onto
the stack and agree to use not the value on the stack, but the value stored at that location.

\pause
In C, \code{&x} is \texttt{x}'s address, and \code{*px} means ``use the value stored at the address \texttt{px}''
\end{frame}
\begin{frame}[fragile]
\frametitle{Pointers}
\label{sec-5_1_11}


A variable that holds an address is called a \emph{pointer}; there's nothing magic about it;  it just happens 
that you can apply the \code{*} operator if you want to use the value it points to.

\pause
\begin{verbatim}
int i = 0;
int *pi = &i;
printf("i = %d, %d\n", i, *pi);
*pi = 10;
printf("i = %d, %d\n", i, *pi);
\end{verbatim}

\pause
If you haven't ensured that a pointer is set to a valid address, you're going to suffer.  If you say
\begin{verbatim}
int *pi;
*pi = 10;
printf("i = %d\n", *pi);
\end{verbatim}
your program may well (and should!) crash.
\end{frame}
\begin{frame}[fragile]
\frametitle{Arrays}
\label{sec-5_1_12}

By definition, given any type \code{type}
\begin{verbatim}
int i;
type p[N];
p[i] == *(p + i);
p + i == &p[i]
\end{verbatim}
\pause
i.e. adding an integer to a pointer gives you an address larger by \code{i*sizeof(type)}
\pause

E.g. \code{float} is usually a 4-byte real, so if \code{p} is \texttt{0xffff0000}, \code{p + 2} is \texttt{0xffff0008}.

\pause
In fact, whenever you refer to an array (\code{p}), it is treated as a pointer to the first element (\code{&p[0]}). 
This is why you can't pass an \texttt{n}-D array to a subroutine.
\end{frame}
\begin{frame}
\frametitle{Strings}
\label{sec-5_1_13}


We introduced \code{char *} as a way of spelling ``string'' and we can now see how it works.

We can write \code{char str[7] = "abcdef";} [5]
\pause

The statement \code{char *str = "abcdef";} is analogous to \code{float *x = &xx;}, 
and \code{*str} is indeed \code{a}.
\pause

The difference between \code{char *str = "abcdef";} and \code{char str[7] = "abcdef";} is where the data is
actually stored; in the former case it's in the \texttt{data} segment, in the latter case it's on the stack.
\end{frame}
\begin{frame}[fragile]
\frametitle{Structs revisited}
\label{sec-5_1_14}


Convinced of the value of a \texttt{struct} such as
\begin{verbatim}
struct Object {
   int id;                              // Object IDs
   float xcen;                          // x-coordinate of centre
   float ycen;                          // y-coordinate of centre
};

typedef struct Object Object;
\end{verbatim}
I wrote a convenience function:
\begin{verbatim}
Object *newObject(const int id, const float xcen, const double ycen) {
   Object obj;

   obj.id = id;
   obj.xcen = xcen;
   obj.ycen = ycen;

   return &obj;
}
#define NOBJECT 1000                    // Maximum number of objects

Object *objs[NOBJECT];                  // our objects

for (int i = 0; i < n; ++i) {
   objs[i] = newObject(i, ...);
}
\end{verbatim}
\end{frame}
\begin{frame}[fragile]
\frametitle{Malloc}
\label{sec-5_1_15}


After some time spent in \texttt{gdb}, I remembered: 

\begin{itemize}
\item Variables in subroutines are irretrievably lost when the routine returns
\end{itemize}
and that's exactly what this does:
\begin{verbatim}
Object *newObject(const int id, const float xcen, const double ycen) {
   Object obj;
   ...
   return @\color{red}\&@obj;
}
\end{verbatim}

\pause
The solution is to get a pointer to a piece of persistent memory.  
C provides this via a call to \code{malloc}:
\begin{verbatim}
#include <stdlib.h>

Object *newObject(const int id, const float xcen, const double ycen) {
   Object @\color{red}*@obj = malloc(sizeof(Object));
   ...
   return obj;
}
\end{verbatim}
\pause
Malloc finds the memory you ask for on the \texttt{heap}.
\end{frame}
\begin{frame}[fragile]
\frametitle{Pointers-to-Pointers}
\label{sec-5_1_16}

How about:
\begin{verbatim}
#define NOBJECT 1000                    // Maximum number of objects

Object *objs[NOBJECT];                  // our objects
\end{verbatim}
\pause

We know that
\begin{verbatim}
int i[10];
\end{verbatim}
and
\begin{verbatim}
int *i;
\end{verbatim}
are equivalent (except for the question of where the data lives).\pause{}
We can provide the needed storage with:
\begin{verbatim}
int *i = malloc(10*sizeof(int));
\end{verbatim}

\pause
So:
\begin{verbatim}
n = ...;
Object **objs = malloc(n*sizeof(Object *)); // our objects
\end{verbatim}
\pause
If you don't know \texttt{n} \emph{a priori}, look up the system call \code{realloc}.
\pause
There is also \code{calloc} but I never use it --- initialising to \texttt{0x0} is a blunt weapon.
\end{frame}
\begin{frame}[fragile]
\frametitle{Free}
\label{sec-5_1_17}

As I said, \code{malloc} returns persistent memory, so it's important to return it to the system
when you've finished with it:
\begin{verbatim}
for (int i = 0; i != n; ++i) {
   free(objects[i]);
}
free(objects);
\end{verbatim}
\pause Failure to do so results in a \emph{memory leak}.

\pause
Freeing a piece of memory more than once usually has catastrophic consequences; don't do it.

\pause
Another failure mode is that sometimes \code{malloc} can't give you the memory you want.  In this case it
returns \texttt{0}, conventionally written \texttt{NULL}.  There is not much you can do when this happens, so a reasonable
response is to abort:
\begin{verbatim}
#include <stdlib.h>
#include <assert.h>

Object *newObject(const int id, const float xcen, const double ycen) {
   Object @\color{red}*@obj = malloc(sizeof(Object));
   assert (obj != NULL);
   ...
   return obj;
}
\end{verbatim}
\end{frame}
\begin{frame}[fragile]
\frametitle{\texttt{struct} and \texttt{typedef}}
\label{sec-5_1_18}

I wrote
\begin{verbatim}
struct Object {
   int id;                              // Object IDs
   ...
};
typedef struct Object Object;
\end{verbatim}
\pause
This can be abbreviated
\begin{verbatim}
typedef struct {
   int id;                              // Object IDs
   ...
} Object;
\end{verbatim}

\pause There are two reasons why you might not always want to omit the name in \code{struct Object \{...\}}:

\begin{itemize}
\item structs can contain pointers to themselves:
\begin{verbatim}
typedef struct List {
    struct List *prev, *next;
    ...;
} List;
\end{verbatim}
\end{itemize}
\pause

\begin{itemize}
\item If you just want to pass an \texttt{Object} around, you can just say \code{struct Object *obj}; the compiler doesn't need to see the details.  \pause This reduces the need to recompile.
\end{itemize}
\end{frame}
\begin{frame}[fragile]
\frametitle{CCD Data again}
\label{sec-5_1_19}


You will recall a code fragment that looked like this:
\begin{verbatim}
U16 data[4096][2048];
...
U16 const peak = data[y][x];            // the (x, y)th pixel
...
void debias(U16 data[][2048], const int nrow, const int ncol);
\end{verbatim}

We are now in a position to do better.
\end{frame}
\begin{frame}
\frametitle{CCD Data again}
\label{sec-5_1_20}

We need a 2-D \texttt{ncol*nrow} array of type \texttt{U16}


\begin{center}
\begin{tabular}{|*{9}{l|}}
\hline
 \null  &     &     &     &     &     &     &     &     \\
\hline
 \null  &     &     &     &     &     &     &     &     \\
\hline
 \null  &     &     &     &     &     &     &     &     \\
\hline
 \null  &     &     &     &     &     &     &     &     \\
\hline
 \null  &     &     &     &     &     &     &     &     \\
\hline
\end{tabular}
\end{center}



\pause Create an 1-D array \texttt{rows} of type \texttt{U16 *} and dimension \texttt{nrow}

\begin{center}
\begin{tabular}{|l|}
\hline
  \\
\hline
  \\
\hline
  \\
\hline
  \\
\hline
  \\
\hline
\end{tabular}
\pause
\begin{tabular}{|l|l|l|l|l|l|l|l|l|l|}
\hline
    &     &    &     &     &     &     &     & \\
\hline
    &     &    &     &     &     &     &     & \\
\hline
    &     &    &     &     &     &     &     & \\
\hline
    &     &    &     &     &     &     &     & \\
\hline
    &     &    &     &     &     &     &     & \\
\hline
\end{tabular}
\end{center}

Make each element of \texttt{rows} point to the start of a row, e.g. \texttt{rows[1]} points to the first pixel
in the second row of data.  Then \texttt{rows[1][2]} is the value of the \texttt{(2, 1)} pixel.
\end{frame}
\begin{frame}[fragile]
\frametitle{CCD Data again}
\label{sec-5_1_21}

\begin{verbatim}
#include <stdlib.h>
#include <stdint.h>
#include <assert.h>
  
typedef uint16_t Pixel_t;

struct Image {
     Pixel_t **rows;
     int nrow, ncol;
};
typedef struct Image Image;

Image *newImage(const int ncol, int const nrow) {
     Image *im = malloc(sizeof(Image));   // the Image
     assert (im != NULL);
     
     im->rows = malloc(nrow*sizeof(Pixel_t *)); // pointers to rows
     assert (im->rows != NULL);
  
     for (int i = 0; i != nrow; ++i) {
        im->rows[i] = malloc(ncol*sizeof(Pixel_t *)); // ith row
        assert (im->rows[i] != NULL);
     }
  
     im->nrow = nrow;
     im->ncol = ncol;
  
     return im;
}
\end{verbatim}
That's a lot of calls to \texttt{malloc}; an alternative is:
\end{frame}
\begin{frame}[fragile]
\frametitle{CCD Data again}
\label{sec-5_1_22}

\begin{verbatim}
#include <stdlib.h>
#include <stdint.h>
#include <assert.h>
  
typedef uint16_t Pixel_t;

struct Image {
     Pixel_t **rows;
     int nrow, ncol;
};
typedef struct Image Image;

Image *newImage(const int ncol, int const nrow) {
     Image *im = malloc(sizeof(Image));   // the Image
     assert (im != NULL);
     
     im->rows = malloc(nrow*sizeof(Pixel_t *)); // pointers to rows
     assert (im->rows != NULL);
  
     im->rows[0] = malloc(nrow*ncol*sizeof(Pixel_t *)); // data
     assert (im->rows[0] != NULL);
  
     for (int i = 0; i != nrow; ++i) {
        im->rows[i] = im->rows[0] + i*ncol;
     }
  
     im->nrow = nrow;
     im->ncol = ncol;
  
     return im;
}
\end{verbatim}
\end{frame}
\begin{frame}[fragile]
\frametitle{Matrices}
\label{sec-5_1_23}

Now we know enough to write ourselves a matrix library
\begin{verbatim}
  typedef struct Matrix {
     float **data;
     int nrow;
     int ncol;
  };
  
  Matrix *newMatrix(int nrow, int ncol);
...
  Matrix *A = newMatrix(10, 10);
  Matrix *B = newMatrix(10, 10);
\end{verbatim}

\pause
However, you can't write:
\begin{verbatim}
Matrix *sum = A + B;
\end{verbatim}
\pause \ldots{} not until you switch to \CPP.








[3] In \CPP you'd probably use a \code{const} variable but scoping rules are different in C, so a macro is appropriate

[4] this isn't quite true;  it could pass a limited number of variables by using \textit{register}s

[5] actually you can omit the dimension;
 it'll be calculated for you. You need to allow an extra character for the \code{'\\0'} that traditionally ends a string in C.
\end{frame}

\end{document}