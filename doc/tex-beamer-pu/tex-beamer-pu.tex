% Created 2011-08-25 Thu 02:55
\documentclass[10pt, t]{beamer}
\usepackage[utf8]{inputenc}
\usepackage[T1]{fontenc}
\usepackage{fixltx2e}
\usepackage{graphicx}
\usepackage{longtable}
\usepackage{float}
\usepackage{wrapfig}
\usepackage{soul}
\usepackage{textcomp}
\usepackage{marvosym}
\usepackage{wasysym}
\usepackage{latexsym}
\usepackage{amssymb}
\usepackage{hyperref}
\tolerance=1000
\usepackage{APC524b}
\providecommand{\alert}[1]{\textbf{#1}}

\title{C}
\author{Robert Lupton\thanks{LooChao@gmail.com}}
\date{23 September 2010}

\begin{document}

\maketitle

\begin{frame}
\frametitle{Outline}
\setcounter{tocdepth}{3}
\tableofcontents
\end{frame}

\section{Introduction}
\label{sec-1}
\subsection{types}
\label{sec-1-1}
\begin{frame}
\frametitle{Introduction}
\label{sec-1-1-1}

C was designed as a system programming language, to remove the necessity of writing operating systems in
assembler.  It's one of the large family of languages deriving from Algol.

\pause
You've seen ``hello world'' before:
CANNOT INCLUDE FILE src/hello.c
\pause

This example comes from ``The C Programming Language'' by Brian Kernighan and Dennis Ritchie (``K\&R'')
\end{frame}
\begin{frame}[fragile]
\frametitle{Declarations}
\label{sec-1-1-2}

 All variables must be declared before they can be used:

\begin{verbatim}
char c = 'a';
char *s = "I am a string";
double d = 1;
float f = 1.0f;
int i = 101010101;
long l = 10;
short s = 10;
\end{verbatim}



\pause
``string'' is spelt \code{char *} for reasons that I'll explain in a bit.
\pause
\code{short} is actually a shorthand for \code{short int} (\code{long} means \code{long int}).
You can also define integral types (\code{char} and \code{int}) as being \code{signed} or \code{unsigned}.
\pause One especially useful qualifier is \code{const}:

\begin{verbatim}
const unsigned long bad = 0xdeadbeef00000000;
\end{verbatim}




\pause An unqualified \code{int} is supposed to be the most efficient integral type on your machine, and is
what you'd generally use unless there was some reason not to.
\end{frame}
\begin{frame}[fragile]
\frametitle{Declaring your own types}
\label{sec-1-1-3}

C allows you to use your own name for a type:

\begin{verbatim}
typedef unsigned short U16;
\end{verbatim}



\pause
Why would I want to do this?

\pause
CCD image data is typically created using a 16-bit A/D converter, so
the natural type for a single pixel is a 2-byte integer.

But C doesn't tell me how large an \code{int} is; I
can find out (using \code{sizeof(int)}) but I don't want to have to change all my declarations when I move
to a new system.

\pause
Actually, these days, I could say:

\begin{verbatim}
#include <stdint.h>
typedef uint16_t U16;
\end{verbatim}



but even that only works if the processor actually has an unsigned 16-bit type.
\end{frame}
\begin{frame}[fragile]
\frametitle{Declarations}
\label{sec-1-1-4}

You can mix code and declarations:

\begin{verbatim}
int pwv = 0;                            // Precipitable Water Vapour
/*
 * Estimate the PWV based on the altitude
 */
...;
/*
 * Given that estimate, find a better value
 */
const int pwv0 = pwv;                   /* initial estimate of pwv */
\end{verbatim}



\pause
The ability to declare variables when they are first needed means that you can usually initialize them too;
when possible, make them \code{const}.
\end{frame}
\begin{frame}[fragile]
\frametitle{Scope}
\label{sec-1-1-5}

A variable's \emph{scope} is the part of the programme it may be referenced from;  in C, a variable's scope
is the nearest set of braces (\code{\{\}}), a \emph{block}.  If it isn't in a block, a variable is visible globally
(i.e. it'll show up when you run \code{nm} on your object file).  If this isn't what you wanted, you can:
 \pause
\begin{itemize}
\item Move it into a function --- remember, global variables should usually be avoided
\end{itemize}
 \pause
\begin{itemize}
\item Label it \code{static} which makes it only visible within the file
\end{itemize}
 \pause
\begin{itemize}
\item Decide that it really \emph{must} be globally visible, and declare it in a header file:

\begin{verbatim}
extern int nread;     // Number of times I've read from a file
\end{verbatim}
\end{itemize}

\pause
It is generally a good idea to declare a variable in as restricted a scope as possible (the ``No Globals'' rule
is a special case of this one).
\end{frame}
\section{Test}
\label{sec-2}
\subsection{test2}
\label{sec-2-1}
\begin{frame}
\frametitle{test}
\label{sec-2-1-1}

1
\end{frame}
\begin{frame}
\frametitle{test}
\label{sec-2-1-2}

2
\end{frame}

\end{document}