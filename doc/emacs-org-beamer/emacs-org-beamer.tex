% Created 2011-12-05 Mon 20:20
\documentclass[bigger, english, 10pt, presentation]{beamer}
\usepackage[utf8]{inputenc}
\usepackage[T1]{fontenc}
\usepackage{fixltx2e}
\usepackage{graphicx}
\usepackage{longtable}
\usepackage{float}
\usepackage{wrapfig}
\usepackage{soul}
\usepackage{textcomp}
\usepackage{marvosym}
\usepackage{wasysym}
\usepackage{latexsym}
\usepackage{amssymb}
\usepackage{hyperref}
\tolerance=1000
\mode<beamer>{\usetheme{CambridgeUS} \usecolortheme{beaver} \setbeamertemplate{items}[ball] \setbeamertemplate{blocks}[rounded][shadow=true] \useoutertheme{infolines}}
\providecommand{\alert}[1]{\textbf{#1}}

\title{Writing Beamer presentations in org-mode}
\author{Eric S Fraga\thanks{e.fraga@ucl.ac.uk}}
\date{2010-03-30 Tue}

\begin{document}

\maketitle

\begin{frame}
\frametitle{Outline}
\setcounter{tocdepth}{3}
\tableofcontents
\end{frame}




\section{Introduction}
\label{sec-1}
\begin{frame}
\frametitle{Overview}
\label{sec-1-1}

\begin{itemize}
\item org-mode template
\item CambridgeUS
\item Berlin
\item beamer structure
\item beamer settings
\end{itemize}
\end{frame}
\section{Methodology}
\label{sec-2}
\begin{frame}
\frametitle{A simple slide}
\label{sec-2-1}

This slide consists of some text with a number of bullet points:
\begin{itemize}
\item the first, very @important@, point!
\item the previous point shows the use of the special markup which
  translates to the Beamer specific \emph{alert} command for highlighting
  text.
\end{itemize}
The above list could be numbered or any other type of list and may
include sub-lists.
\end{frame}
\begin{frame}
\frametitle{A more complex slide}
\label{sec-2-2}

This slide illustrates the use of Beamer blocks.  The following text,
with its own headline, is displayed in a block:
\begin{theorem}[Org mode increases productivity]
\label{sec-2-2-1}

\begin{itemize}
\item org mode means not having to remember \LaTeX commands.
\item it is based on ascii text which is inherently portable.
\item Emacs!
\end{itemize}

    \hfill \(\qed\)
\end{theorem}
\end{frame}
\begin{frame}
\frametitle{Two columns}
\label{sec-2-3}
\begin{columns}
\begin{column}{0.4\textwidth}
%% A block
\label{sec-2-3-1}

\begin{itemize}
\item this slide consists of two columns
\item the first (left) column has no heading and consists of text
\item the second (right) column has an image and is enclosed in an
      @example@ block
\end{itemize}
\end{column}
\begin{column}{0.6\textwidth}
\begin{example}[A screenshot]
\label{sec-2-3-2}
\end{example}
\end{column}
\end{columns}
\end{frame}
\begin{frame}[fragile,t]
\frametitle{Babel}
\label{sec-2-4}
\begin{columns}
\begin{column}{0.45\textwidth}
\begin{block}{Octave code}
\label{sec-2-4-1}


\begin{verbatim}
A = [1 2 ; 3 4]
b = [1; 1];
x = A\b
\end{verbatim}
\end{block}
\end{column}
\begin{column}{0.4\textwidth}
\begin{block}<2->{The output}
\label{sec-2-4-2}



\begin{verbatim}
A =

   1   2
   3   4

x =

  -1
   1
\end{verbatim}
\end{block}
\end{column}
\end{columns}
\end{frame}
\section{Conclusions}
\label{sec-3}
\begin{frame}
\frametitle{Summary}
\label{sec-3-1}

\begin{itemize}
\item org is an incredible tool for time management
\item @but@ it is also excellent for writing and for preparing presentations
\item Beamer is a very powerful \LaTeX{} package for presentations
\item the combination is unbeatable!
\end{itemize}
\end{frame}

\end{document}