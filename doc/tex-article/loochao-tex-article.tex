\documentclass[preprint,preprintnumbers]{revtex4}
\usepackage{amssymb}
\usepackage{amsmath}
\usepackage{graphicx}
\usepackage{epsfig}
\usepackage{dcolumn}
\usepackage{bm}

\begin{document}

\title{ loochao tex article }
\author{Chao Lu}
\affiliation{Princeton University, Princeton, NJ 08544-5263}

\begin{abstract}
abstract goes in here
\end{abstract}

\maketitle

\section{Section1}

\appendix

\section{Appendix}

\begin{thebibliography}{99}
\bibitem{Chao84} Chao LU and LooChao, Phys. Rev. Lett. \textbf{77},
  5206 (1996).
\end{thebibliography}

\end{document}
