\usepackage[utf8]{inputenc}
\usepackage[T1]{fontenc}
\usepackage{fixltx2e}
\usepackage{graphicx}
\usepackage{longtable}
\usepackage{float}
\usepackage{wrapfig}
\usepackage{soul}
\usepackage{textcomp}
\usepackage{marvosym}
\usepackage{wasysym}
\usepackage{latexsym}
\usepackage{amssymb}
\usepackage{hyperref}
\usepackage{loochao}

\title{Split slides}
\author{Chao LU}
\date{}

\title{Title here}
\author{Some author}
\date{}

\begin{document}

\begin{frame}
  \titlepage
\end{frame}

\note[itemize]{
\item The audience sees the slides, but you see your notes.
  \bigskip
\item Or, if your don't have notes, you can use mirror mode.
}

\section{One section}
\begin{frame}
  \frametitle{One frame}
\end{frame}

\begin{frame}{Another Slide}
  \begin{itemize}
  \item Here is another slide.
  \item This one doesn't have a note.
  \item Slides without a note, are always shown on both displays.
  \end{itemize}
\end{frame}

Some text included only in the article mode.

\begin{frame}{Slides vs.{} Notes}
  Your slides can be in one of three modes:
  \begin{itemize}
  \item Just slides, no notes
  \item Slides and notes interleaved
  \item Superwide pages: slides and notes side-by-side
  \end{itemize}
\end{frame}

\note[itemize]{
\item Just slides, no notes
  \begin{itemize}
  \item For presentations without any notes
  \end{itemize}
\item Slides and notes interleaved
  \begin{itemize}
  \item BEAMER's .nav file is used to determine if a particular page is a slide or a note.
  \item The .nav file can be either embedded into the PDF (like it is in this one) or next to it in the same directory
  \end{itemize}
\item Superwide pages: slides and notes side-by-side
  \begin{itemize}
  \item For this mode, include the \texttt{\footnotesize pgfpages} package and change the BEAMER option ``\texttt{\footnotesize show notes}'' to ``\texttt{\footnotesize show notes on second screen}''.
  \end{itemize}
}

\end{document}
